\documentclass[a4paper, 12pt]{msc-style}
\usepackage{msc}

\begin{document}

\setlength{\columnsep}{0cm}
\renewcommand{\thefootnote}{\fnsymbol{footnote}}

\newcommand{\myheader}[1]{
  \begin{tabular}{p{1in}p{1in}r}
    \parbox{3cm}{\includegraphics[height=30mm]{cimer}} & &
    \parbox{7.0cm}{
      \begin{center}
        #1
      \end{center}
    }
  \end{tabular}
}

\newcommand{\huntitle}{Webalkalmazások tesztelése Erlangban}

\newcommand{\huninstitute}{
  Eötvös Loránd Tudományegyetem\\
  Informatikai Kar\\
  Programozási Nyelvek és Fordítóprogramok Tanszék
}

\newcommand{\hunsupervisors}{
  {\bfseries Horpácsi Dániel}\\
  Programozási Nyelvek és \\Fordítóprogramok Tanszék \\
  Eötvös Loránd Tudományegyetem
}
\newcommand{\hunauthor}{
  {\bf Bodnár István}\\
  Programtervező informatikus\\
  nappali tagozat
}

\begin{titlepage}\thispagestyle{empty}
  \myheader{\huninstitute}
  \vspace{1cm}\hrule\vspace{2cm}
  \begin{center}
    {\LARGE \huntitle }\\
    \vspace{4cm}
    \begin{tabular}{lr}
      \begin{minipage}{7cm}
        \begin{tabular}[b]{c}
          \emph{Témavezető:}
          \\\\
          \hunsupervisors
        \end{tabular}
      \end{minipage}
      &
      \raisebox{0.32cm}{
        \begin{minipage}{7cm}
          \begin{tabular}[b]{c}
            \emph{Szerző:}
            \\\\
            \hunauthor
          \end{tabular}
        \end{minipage}
      }
    \end{tabular}
    \vfill
    Budapest, 2014.
  \end{center}
\end{titlepage}

\cleardoublepage
\tableofcontents
\thispagestyle{empty}
\cleardoublepage

\chapter{Bevezető}
Napjainkban fontos szerepet tölt be a szoftverfejlesztés folyamatában az elkészült
termék megfelelő tesztelése, ezáltal a minőség javítása.
A szoftverfejlesztési módszerek közül egyre nagyobb szerepet kap az agilis szoftverfejlesztés, ahol szükségessé válik a tesztautomatizáció. 
Ennek segítségével gyorsabbá válik a tesztelés az által, hogy emberek helyett egy eszköz hajtja végre az előre meghatározott teszteket.
Ezek a tesztek többfélék lehetek attól függően, hogy mit szeretnénk tesztelni, az alkalmazás funkcionalitását (funkcionális tesztek), 
az alkalmazás részeinek megfelelő működését (unit tesztek) vagy a rendszer működését új funkciók hozzáadása után (regressziós tesztek).\\\\
Webalkalmazások teszteléséhez kényelmes megoldást jelent a Selenium\cite{Selenium}.

\chapter{X Y Z}

\section{A B C}

\subsection{A S D}

x y z

\cleardoublepage
{
  \singlespacing
  \bibliographystyle{abbrv}
  \bibliography{bibfile}
}

%\appendix

\end{document}
